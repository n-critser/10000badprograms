% Created 2019-05-19 Sun 15:04
\documentclass[11pt]{article}
\usepackage[utf8]{inputenc}
\usepackage[T1]{fontenc}
\usepackage{fixltx2e}
\usepackage{graphicx}
\usepackage{longtable}
\usepackage{float}
\usepackage{wrapfig}
\usepackage{rotating}
\usepackage[normalem]{ulem}
\usepackage{amsmath}
\usepackage{textcomp}
\usepackage{marvosym}
\usepackage{wasysym}
\usepackage{amssymb}
\usepackage{hyperref}
\tolerance=1000
\usepackage[margin=3cm]{geometry}
\usepackage{url}
\usepackage{listings}
\usepackage{amsmath}
\usepackage{amssymb}
\lstset{language=C}
\usepackage{geometry}
\usepackage{tikz}
\usetikzlibrary{shapes,arrows,automata}
\usepackage[all]{xy}
\geometry{left=3.5cm,top=1.0cm,right=1.5cm,bottom=1.0cm,marginparsep=7pt, marginparwidth=.6in}
\author{N.Critser}
\date{\today}
\title{numnums}
\hypersetup{
  pdfkeywords={},
  pdfsubject={},
  pdfcreator={Emacs 24.5.1 (Org mode 8.2.10)}}
\begin{document}

\maketitle

\section*{congruence}
\label{sec-1}
\section*{primality}
\label{sec-2}
\section*{divisability}
\label{sec-3}
\section*{basis-representation}
\label{sec-4}
\begin{itemize}
\item THEOREM:
\label{sec-4-0-1}

Let k be any integer larger than 1. Then, for each positive integer n, there exists a representation
\\
n = a$_{\text{0}}$k$^{\text{s}}$ + a$_{\text{1}}$k$^{\text{s-1}}$ + . . . + a$_{\text{s}}$ 
\\
where a$_{\text{0}}$ $\ne$ 0, and where each a$_{\text{i}}$ is nonnegative and less than k.
\\
This representation of n is unique.
\\
It is known as the representation of n to the base k.

\item PROOF:
\label{sec-4-0-2}
Let b$_{\text{k}}$(n) be the number of representations of n to the base k.
We need to show that b$_{\text{k}}$(n) = 1 and only 1.
\\
\\
Some of the coefficients  a$_{\text{i}}$, can be equal to zero for a particular
representation of n. But that doesn't effect the representation,so we will
exclude terms that are zero.
\\
suppose : n = a$_{\text{0}}$k$^{\text{s}}$ + a$_{\text{1}}$k$^{\text{s-1}}$ + . . . + a$_{\text{s-t}}$k$^{\text{t}}$,
\\ 
neither a$_{\text{0}}$ or a$_{\text{s-t}}$ = 0. So subtracting 1 from n gives,
\\
n-1 = a$_{\text{0}}$k$^{\text{s}}$ + a$_{\text{1}}$k$^{\text{s-1}}$ + . . . + a$_{\text{s-t}}$k$^{\text{t}}$-1
\\
n-1 = a$_{\text{0}}$k$^{\text{s}}$ + a$_{\text{1}}$k$^{\text{s-1}}$ + . . . + (a$_{\text{s-t}}$-1)k$^{\text{t}}$ + k$^{\text{t}}$ - 1
\\
Since $\sum$$_{\text{j=0}}^{\text{n-1}}$x$^{\text{j}}$  =  \frac\{x$^{\text{n-1}}$\}\{n-1\}
\\
n-1 = a$_{\text{0}}$k$^{\text{s}}$ + a$_{\text{1}}$k$^{\text{s-1}}$ + . . . + (a$_{\text{s-t}}$-1)k$^{\text{t}}$ + $\sum$$_{\text{j=0}}^{\text{t-1}}$(k-1)k$^{\text{j}}$
\end{itemize}
\section*{fundatmental-theorem-of-arithmatic}
\label{sec-5}
% Emacs 24.5.1 (Org mode 8.2.10)
\end{document}